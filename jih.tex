% Options for packages loaded elsewhere
\PassOptionsToPackage{unicode}{hyperref}
\PassOptionsToPackage{hyphens}{url}
%
\documentclass[
  12pt,
  norsk,
]{article}
\usepackage{lmodern}
\usepackage{amssymb,amsmath}
\usepackage{ifxetex,ifluatex}
\ifnum 0\ifxetex 1\fi\ifluatex 1\fi=0 % if pdftex
  \usepackage[T1]{fontenc}
  \usepackage[utf8]{inputenc}
  \usepackage{textcomp} % provide euro and other symbols
\else % if luatex or xetex
  \usepackage{unicode-math}
  \defaultfontfeatures{Scale=MatchLowercase}
  \defaultfontfeatures[\rmfamily]{Ligatures=TeX,Scale=1}
\fi
% Use upquote if available, for straight quotes in verbatim environments
\IfFileExists{upquote.sty}{\usepackage{upquote}}{}
\IfFileExists{microtype.sty}{% use microtype if available
  \usepackage[]{microtype}
  \UseMicrotypeSet[protrusion]{basicmath} % disable protrusion for tt fonts
}{}
\makeatletter
\@ifundefined{KOMAClassName}{% if non-KOMA class
  \IfFileExists{parskip.sty}{%
    \usepackage{parskip}
  }{% else
    \setlength{\parindent}{0pt}
    \setlength{\parskip}{6pt plus 2pt minus 1pt}}
}{% if KOMA class
  \KOMAoptions{parskip=half}}
\makeatother
\usepackage{xcolor}
\IfFileExists{xurl.sty}{\usepackage{xurl}}{} % add URL line breaks if available
\IfFileExists{bookmark.sty}{\usepackage{bookmark}}{\usepackage{hyperref}}
\hypersetup{
  pdftitle={Assignment 1 Data Science},
  pdfauthor={Sofie Brynjelsen; Silje Marie Danielsen},
  pdflang={nb-no},
  hidelinks,
  pdfcreator={LaTeX via pandoc}}
\urlstyle{same} % disable monospaced font for URLs
\usepackage[margin=1in]{geometry}
\usepackage{graphicx,grffile}
\makeatletter
\def\maxwidth{\ifdim\Gin@nat@width>\linewidth\linewidth\else\Gin@nat@width\fi}
\def\maxheight{\ifdim\Gin@nat@height>\textheight\textheight\else\Gin@nat@height\fi}
\makeatother
% Scale images if necessary, so that they will not overflow the page
% margins by default, and it is still possible to overwrite the defaults
% using explicit options in \includegraphics[width, height, ...]{}
\setkeys{Gin}{width=\maxwidth,height=\maxheight,keepaspectratio}
% Set default figure placement to htbp
\makeatletter
\def\fps@figure{htbp}
\makeatother
\setlength{\emergencystretch}{3em} % prevent overfull lines
\providecommand{\tightlist}{%
  \setlength{\itemsep}{0pt}\setlength{\parskip}{0pt}}
\setcounter{secnumdepth}{-\maxdimen} % remove section numbering
\ifxetex
  % Load polyglossia as late as possible: uses bidi with RTL langages (e.g. Hebrew, Arabic)
  \usepackage{polyglossia}
  \setmainlanguage[]{norsk}
\else
  \usepackage[shorthands=off,main=norsk]{babel}
\fi

\title{Assignment 1 Data Science}
\usepackage{etoolbox}
\makeatletter
\providecommand{\subtitle}[1]{% add subtitle to \maketitle
  \apptocmd{\@title}{\par {\large #1 \par}}{}{}
}
\makeatother
\subtitle{Reproduserbarhet}
\author{Sofie Brynjelsen \and Silje Marie Danielsen}
\date{}

\begin{document}
\maketitle

\newpage

\hypertarget{innledning}{%
\section{Innledning}\label{innledning}}

I denne oppgaven skal vi ta for oss hva reproduserbarhet og
replikerbarhet er. Samt problemet med og reprodusering.

I dagens samfunn er vitenskapen i utvikling, og det vil være et stort
behov for at man har tillit til forskningen som blir gjort. For å skape
tillit er det vesentlig at funnen som er blitt gjort, kan reproduseres.
Det vil si at når en gjennomfører samme analyse basert på samme data,
vil man endre opp med de eksakt samme svarene. I forbindelse med
reproduserbarhet vil det være vesentlig å nevne replikerbarhet.
Replikerbarhet handler om at man får samme konklusjon, når man
gjennomfører samme undersøkelse med nye data. Dette omtales også som
gullstandarden når det gjelder vitenskap. vi kan se på reproduserbarhet
som nødvendig, men ikke tilstrekkelig for replikerbarhet.

Forskere har funnet ut at det til tider er vanskelig med
reproduserbarhet. I denne forbindelse nevner (Peng 2011) at
reproduserbarhet bær være et minstekrav for at en artikkel skal
produseres.

Robus og pålitelig forskning er grunnlaget for vitesnakpelig utvikling
og fremgang. Dette avhenger av at forskernes evne til å samle inn data
fra tidligere arbeid. ``robust and reliable sciene'' omhandler at
forskningen skal være reproduserbar, replikerbar og generaliserbar. Når
det kommer til defineringen av disse uttrykkene, har det oppstått
forvirring og misforståelser. I denne fobindelse vil vi velge å definere
disse på engelsk. (Bollen mfl. 2015)

\hypertarget{definisjoner}{%
\subsection{Definisjoner}\label{definisjoner}}

\emph{"\textbf{Replicability} refers to: the ability of a researcher to
duplicate the results of a prior study if the same procedures are
followed but the new data are collected.}"

\emph{"\textbf{Generalizability} refers to: wheter the results of a
study apply in other contexts or populations that differ from the
originals.}"

\emph{"\textbf{Reproducibility} refers to: the ability of a researcher
to duplicate the results of a prior study using the same materials and
prcedures used by the original investigator.}"

\emph{\emph{Reproduserbarhet}} kan også deles inn i tre hovedkategorier;
\emph{methods reproducibility}, \emph{results reproducibility} og
\emph{robustness and generalizability}.

Methods reproducibility: ``Methods reproducibility refers to the
provision of enough detail about study procedures and data so the same
procedures could, in theory or in actuality, be exactly repeated''

Results reproducibility: ``Results reproducibility (previously descrived
as replicability) refers to obtaining the same results from the conduct
of an independent study whose procedure are as closely matched to the
original experiment as possible.''

Robustness and generallzability: ``We briefly introduce these terms
because they are sometimes used in lieu of the term reproducibility.
Robustness refers to the stability of experimental conclusjon to
varaitions in either baseline assumptions or experimental procedures. It
is somewhat related to the concept of generalizability (also known as
transportability), which refers to the persistence of an effect in
settings different from and outside of an experimental framework''.

\newpage

\#\#Problemer Hoveddel

\hypertarget{publication-bias}{%
\section{\texorpdfstring{\textbf{Publication
bias}}{Publication bias}}\label{publication-bias}}

Et annet problem knyttet til forskning er publiserings skeivhet også
kalt for publication bias. Dette omhandler at sannsynligheten for at en
studie skal bli publisert avhenger av konklusjonen til studiet. Etter å
ha undersøkt er det mange som har funnet ut at det er veldig få studier
som har en negativ konklusjon. Det vil si at man ikke forkaster vår null
hypotese. dette kan være et problem knyttet til at forskningen kan vise
at det ikke har noen effekt, noe som videre kan fære til at man går
glipp av viktig kunnskap. Et ``worst case scenario'' er at de
vitenskapelige skriftene blir en stor samling av type 1 error.

\hypertarget{type-1-error}{%
\section{\texorpdfstring{\textbf{Type 1
error}}{Type 1 error}}\label{type-1-error}}

Type 1 error vil si at man forkaster H0 når den i virkeligheten skal
beholdes. Et av problemene knyttet til dette tema er hvis man for
eksempel ufører like 100 studier som finner ingen effekt, og vi har en
som finner en effekt, vil det være en sjanse for at man finner en studie
i litteraturen som viser et feil resultat. Dette blir ofte omtalt som
``the File Drawer Problem'' (Rosenthal 1979). Det vil si at de studiene
som viser at man ikke kan forkaste H0, ikke blir publiserte. Studier som
viser at man kan forkaste H0, blir publiserte. (Simmons, Nelson, og
Simonsohn 2011) hevder at dette er en svært kostbar feil å gjøre, da de
blir liggende i literaturen lenge. Dette vil også medføre at det blir
mindre muligheter for andre å reprodusere studiene for å se om de
samsvarer. Falske resultater vil overleve over en lang periode. et annet
problem knyttet til dette er at man kan bruke de falske positve svarene
som utgangspunkt for nye undersøkelser. Man vil da bruke store ressurser
på å arbeide med noe som ikke er riktig. Videre kan dette føre til
politiske følger og kostbare reformer som da i sin helhet er begrunnet
ut ifra vitenskap som ikke gjelder. Alt i alt vil dette svekke
troverdigheten.

\hypertarget{publication-bias-and-meta-analysis}{%
\section{\texorpdfstring{\textbf{Publication bias and
meta-analysis}}{Publication bias and meta-analysis}}\label{publication-bias-and-meta-analysis}}

publiserings skeivhet kan videre forplante seg i metaanalyser, hvor man
tar for seg mange artikler innenfor et fagområde. hensikteten med
meta-analysis er å finne det generelle svaret ved å sammenfatte
resultatene fra den tidligere forskningen. Dersom man da bare tar med de
falske positive artiklene som blir publisert, vil dette ha stor
betydning for hvilen konklusjone man ender med.(Young, Ioannidis, og
Al-Ubaydli 2008)

\hypertarget{the-replication-crisis}{%
\section{\texorpdfstring{\textbf{The replication
crisis}}{The replication crisis}}\label{the-replication-crisis}}

Et annet problem som angår dette tema er det som blir kalt for the
replication crisis. Dette har sine røtter innen psykologi men etter
hvert er det flere fagområder, som for eksmepel økonomi som har vist
interesse for tema. (Simmons, Nelson, og Simonsohn 2011) beviste at man
vil føle seg yngre ved å høre på ``When I'm sixty four'' av The beatles.
Dette er forskning som ikke stemmer, og etter dette utfallet ville de
finne en løsning på problemet når det kommer til publisering av
forskning. De innførte dermed seks ulike krav til forfatterne og fire
guidelines til redaktørene. Hovedformålet med disse kravene og
guidlinesene er at man skal kunne unngå å få slike resultat, som den
tidligere nevnte forskningen.

\newpage

\hypertarget{avslutning}{%
\section{Avslutning}\label{avslutning}}

\hypertarget{terminal-window}{%
\section{Terminal window}\label{terminal-window}}

Under her kommer en Git instruksjonsliste fra Jørn Grolemund og Wickham
(udatert)

\begin{enumerate}
\def\labelenumi{\arabic{enumi}.}
\tightlist
\item
  git clone \url{https://github.com/SBrynjelsen/Masterr}
\item
  git branch jih
\item
  git checkout jih
\item
  Gjør endringer
\item
  git add --all
\item
  git commit -m ``første commit jih''
\item
  git push --set-upstream origin jih
\item
  git push
\end{enumerate}

\hypertarget{referanser}{%
\subsection{Referanser}\label{referanser}}

\hypertarget{refs}{}
\leavevmode\hypertarget{ref-bollen2015}{}%
Bollen, Kenneth, John T. Cacioppo, Jon A. Krosnick, James L. Olds, og
Robert M. Kaplan. 2015. «Social, Behavioral, and Economic Sciences
Perspectives on Robust and Reliable Science». Report of the Subcommittee
on Replicability in Science Advisory Committee to the National Science
Foundation Directorate for Social, Behavioral, and Economic Sciences.
NSF.

\leavevmode\hypertarget{ref-grolemund}{}%
Grolemund, Garrett, og Hadley Wickham. udatert. \emph{R for Data
Science}.

\leavevmode\hypertarget{ref-peng2011}{}%
Peng, Roger D. 2011. «Reproducible Research in Computational Science».
\emph{Science} 334 (6060): 1226--7.
\url{https://doi.org/10.1126/science.1213847}.

\leavevmode\hypertarget{ref-rosenthal1979}{}%
Rosenthal, R. 1979. «The File Drawer Problem and Tolerance for Null
Results.» I, 86:638--41. Psychological Bulletin.

\leavevmode\hypertarget{ref-simmons2011}{}%
Simmons, Joseph P., Leif D. Nelson, og Uri Simonsohn. 2011.
«False-Positive Psychology: Undisclosed Flexibility in Data Collection
and Analysis Allows Presenting Anything as Significant».
\emph{Psychological Science} 22 (11): 1359--66.
\url{https://doi.org/10.1177/0956797611417632}.

\leavevmode\hypertarget{ref-young2008}{}%
Young, Neal S, John P. A Ioannidis, og Omar Al-Ubaydli. 2008. «Why
Current Publication Practices May Distort Science». \emph{PLoS Medicine}
5 (10): e201. \url{https://doi.org/10.1371/journal.pmed.0050201}.

\hypertarget{appendiks}{%
\subsection{Appendiks}\label{appendiks}}

\end{document}
